\documentclass{article}

\usepackage{geometry} % пакет для установки полей
\geometry{top=1cm} % отступ сверху
\geometry{bottom=2cm} % отступ снизу
\geometry{left=1cm} % отступ справа
\geometry{right=1cm} % отступ слева

\usepackage[T2A]{fontenc}
\usepackage[utf8]{inputenc}
\usepackage[english,russian]{babel}

\usepackage{amsmath}
\usepackage{amsthm}
\usepackage{amssymb}
\usepackage{graphicx}
\usepackage{asymptote}

\theoremstyle{plain}
\newtheorem{theorem}{Теорема}[section]
\newtheorem{algorithm}{Алгоритм}[section]
\newtheorem{statement}{Утверждение}[section]
\newtheorem{corollary}{Следствие}[theorem]

\theoremstyle{definition}
\newtheorem{definition}{Определение}[section]

\renewcommand{\proofname}{Доказательство}

\newcommand{\plain}{\mathbb{R}^2}
\newcommand{\eqdef}{\stackrel{\mathrm{def}}{=}}

\pagestyle{empty}

\begin{document}

\begin{enumerate}
\item \begin{statement} Доказать или опровергнуть, что любое двоичное дерево является двойственным графом триангуляции некоторого полигона. 
\begin{proof} 
Будем доказывать по индукции, что двоичное дерево является двойственным графом триангуляции. \\
\textbf{База:} Один треугольник соответствует одной вершине. \\
\textbf{Индукционное предположение} Пусть бинарное дерево с n-1 является двойствнным графом триангуляции полигона \\
\textbf{Индукционный переход} $n-1 \rightarrow n$. (Считаем что мы берем любой соседний треугольник к треугольнику находящимуся в дереве)
Так как треугольник в триангулции имеет только трех соседей. Следовательно новый треугольник смежен с любым треугольником(вершиной) у которых степень равна 1, 2. Понятно что добавление этой вершины в бинарное дерево не нарушит его свойств. \\ 
Отдельно рассмотрим ситуацию, когда корень имеет двух детей. И к нам приходит треугольник, являющийся соседом "корневому" треугольнику. 
В таком случае у нас корнем страновится новый пришедший треугольник и мы не нарушаем свойств бинарности. Ну и в конечном итоге такой треугольник у которого степень вершины будет равна единице или двум найдется так как треугольники у которых ребра входят в границу полигона имеют строго меньше трех соседей.
\end{proof}
\end{statement}

\item \textbf{ Непланарность графа $K_5$} 
\begin{statement} Граф $K_5$ - полный пятивершинный граф. Доказать, что он не может быть уложен на плоскость прямолинейным способом.  
\begin{proof} 
Воспользуемся формулой Эйлера для графа В - P + Г = 2, где В-количество вершин, Р - количество ребер, Г - количество граней. 
Для $K_5$ - В = 5, Р = 10, тогда граней Г = 7. 
Понятно что каждая грань имеет хотя бы три ребра, тогда $3*7 \leq 2*10$
Получаем противоречие, следовательно граф не планарен.
\end{proof}
\end{statement}

\item \textbf{Извлечение выпуклой оболочки из триангуляции множества точек}
Пусть  некоторая триангуляция множества точек храниться следующим образом. Для каждой вершины хранится указатель на какой-нибудь инцидентный ей треугольник, а для каждого треугольника хранятся указатели на инцидентные ему вершины в порядке обхода против часовой стрелки и указатели на смежные треугольники. Причем i-й $(i \in {1, 2, 3}) $ треугольник граничит с данными по ребру противолежащему i-й вершине. Если соседа по i-му ребру нет, указатель равен nil. Придумать алгоритм, перечисляющий вершины выпуклой оболочки в порядке обхода против часовой стрелки \\
Я сделал вывод, раз для каждой вершины храниться указатель на какой-нибудь инцидентный ей треугольник, то значит есть где-то массив вершин и за O(n) мы можем достать крайную левую вершину. Она точно будет принадлежать ответу. \\   
\textbf{Алгоритм}\\
\begin{enumerate}
\item $startPoint \leftarrow getMinLeftPoint()$
\item $currentPoint \leftarrow startPoint$
\item do $\{$ 
\item    $triangle \leftarrow getIncidentTriangle()$
\item    $nextCurrentPoint \leftarrow triangle.nextPoint(currentPoint)$	
\item    $currentTriangle \leftarrow triangle$
\item    $while(currentTriangle.getNeighbors(Edge(currentPoint, nextCurrentPoint)) != nil) \{$
\item         $currentTriangle \leftarrow currentTriangle.getNeighbors(Edge(currentPoint, nextCurrentPoint))$
\item         $nextCurrentPoint \leftarrow currentTriangle.nextPoint(currentPoint)$
\item    $\}$
\item    $currentPoint \leftarrow currentTriangle.nextPoint(currentPoint)$
\item    $answer \leftarrow currentPoint$
\item $\}$ while(currentPoint != startPoint)
\end{enumerate}
Функция nextPoint(currentPoint) - возвращает следующую точку после currentPoint в обходе против часовой стрелки. \\
Функция getNeighbors(Edge) - возвращает соседа, соответствующие данному ребру.
Функция getIncidentTriangle() - возвращает инцидентный треугольник данной точки.

\item \textbf{Единственность триангуляции Делоне} 
\begin{statement} Доказать, что если никакие четыре точки из множества не лежат на одной окружности, то триангуляция Делоне этого множества точек единственна.  
\begin{proof} 
Пусть это не так, то есть существует несколько триангуляций Делоне, при том что любые четыре точки не лежат на одной окружности. 
Рассмотрим четыре точки $\{1, 2, 3, 4\}$, в которых отличаются две триангуляций. В первой триангуляции есть ребро 1-3, а во второй 2-4. 
Понятно, что треугольники проходящие через эти точки в триангуляции получаются после Flip-ов треугольников(в-первом случае треугольники: 1-2-3, 1-3-4, а во-втором cслучае: 1-2-4, 4-3-2). В итоге мы получили, что есть пары треугольников для которых выполняется критерий Делоне и один случай приводиться к другому через операцию Flip. С учетом того что у нас никакие четыре точки не лежат на окружности получаем противоречие с операцией Flip. 
\end{proof}
\end{statement}

\item \textbf{Поиск ближайщей вершины} 
\begin{statement} Пусть для некоторого множества точек построена триангуляция Делоне. Доказать или опровергнуть, что ближайщей к произвольное точке плоскости будет одна из вершин треугольника, в котором лежит данная точка.      
\begin{proof} 
\end{proof}
\end{statement}

\end{enumerate}
\end{document}
