\documentclass{article}

\usepackage{geometry} % пакет для установки полей
\geometry{top=1cm} % отступ сверху
\geometry{bottom=2cm} % отступ снизу
\geometry{left=1cm} % отступ справа
\geometry{right=1cm} % отступ слева

\usepackage[T2A]{fontenc}
\usepackage[utf8]{inputenc}
\usepackage[english,russian]{babel}

\usepackage{amsmath}
\usepackage{amsthm}
\usepackage{amssymb}
\usepackage{graphicx}
\usepackage{asymptote}

\theoremstyle{plain}
\newtheorem{theorem}{Теорема}[section]
\newtheorem{statement}{Утверждение}[section]
\newtheorem{corollary}{Следствие}[theorem]

\theoremstyle{definition}
\newtheorem{definition}{Определение}[section]

\renewcommand{\proofname}{Доказательство}

\newcommand{\plain}{\mathbb{R}^2}
\newcommand{\eqdef}{\stackrel{\mathrm{def}}{=}}

\pagestyle{empty}

\begin{document}
\begin{statement}
Граф - должен быть связен
\begin{proof}
В доказательстве я использую, что исходя из построения, грани в графе означают ячейки в диаграмме Вороного. \\ 
%Каждая ячейка является простым полигоном. Граница полигона образует замкнутую цепь в графе. Замкнутая цепь является циклом.
Пусть граф не связен. Тогда есть несколько компонент связности. Выберем две соседние компноненты связности. Найдем такие две такие ячейки ($def f_1, f_2$) из разных компонент связности у которых расстояние минимально (расстояние между этими сайтами - минимально). Так как компоненты связности соседние, между этими двумя ячейками ничего нет (сайтов тоже нет). Получаем противоречие с определением ячейки Вороного.    
\end{proof}
\end{statement}

\end{document}
