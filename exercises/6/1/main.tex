\documentclass{article}

\usepackage{geometry} % пакет для установки полей
\geometry{top=1cm} % отступ сверху
\geometry{bottom=2cm} % отступ снизу
\geometry{left=1cm} % отступ справа
\geometry{right=1cm} % отступ слева

\usepackage[T2A]{fontenc}
\usepackage[utf8]{inputenc}
\usepackage[english,russian]{babel}

\usepackage{amsmath}
\usepackage{amsthm}
\usepackage{amssymb}
\usepackage{graphicx}
\usepackage{asymptote}

\theoremstyle{plain}
\newtheorem{theorem}{Теорема}[section]
\newtheorem{statement}{Утверждение}[section]
\newtheorem{corollary}{Следствие}[theorem]

\theoremstyle{definition}
\newtheorem{definition}{Определение}[section]

\renewcommand{\proofname}{Доказательство}

\newcommand{\plain}{\mathbb{R}^2}
\newcommand{\eqdef}{\stackrel{\mathrm{def}}{=}}

\pagestyle{empty}

\begin{document}

\begin{figure}
\begin{center}
\begin{asy}
unitsize(0.5inch);
pair a = (0, 0);
pair b = (-1, 1); 
pair c = (1, -1);
draw(a--b);
draw(b--c);
dot(a);
dot(b);
dot(c);
label("$v_2$", a, S);
label("$v_1$", b, S); 
label("$v_3$", c, S);
\end{asy}
\end{center}
\end{figure}

\begin{statement}
Каждая вершина графа имеет степень $\geq 3$
\begin{proof}
\begin{enumerate}
\item Степень вершины $\neq 0$ \\
Д-во: Граф связен.
\item Степень вершины $\neq 1$ \\
Д-во: Так как каждая ячейка - это простой полигон. Граф построенный по границе простого полигона является замкнутой цепью. А в замкнутой цепи нет вершин, степень которых $\leq 1$
\item Степень вершины $\neq 2$ \\
Д-во: Тогда два ребра ($v_1v_2$ и $v_2v_3$), исходящие из этой вершины, принадлежат двум ячейкам Вороного. Исходя из определения ячейки и метода построения этого графа (ребрами являются отрезки, прямые, лучи), следует, что эти ребра лежат на одной прямой. Тогда смысла в этой вершине нет и ее можно выбросить. Следовательно в графе нет вершин степени два. 
\end{enumerate}
\end{proof}
\end{statement}
\end{document}
